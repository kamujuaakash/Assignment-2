\documentclass[journal,12pt,twocolumn]{IEEEtran}
\usepackage[utf8]{inputenc}
\usepackage{amssymb,amsmath,mathtools} 
\usepackage{amsfonts}  
\usepackage{graphicx}  
\usepackage{times}
\usepackage{gensymb}
\usepackage{algorithm2e}
\usepackage{circuitikz}
\graphicspath{{images/}}
\DeclareGraphicsExtensions{.pdf,.eps,.ps,.png,.jpg,.jpeg}

\title{ AI1110 Assignment 2}
\author{AI21BTECH11001 }
\date{April 2022}

\begin{document}

\maketitle

\begin{center}
    ICSE Grade 12th 2017
\end{center}

\textbf{Question 3(b) :} If A,B and C are the elements of Boolean Algebra,simplify the expression\\
$(A^\prime + B^\prime)(A + C^\prime) + B^\prime(B + C)$. 
Draw the simplified expression

\textbf{Solution:}
\begin{align}
 =& (A^\prime + B^\prime)(A + C^\prime) + B^\prime(B + C)\\
 =& A^\prime(A + C^\prime) + B^\prime(A + C^\prime)+ B^\prime B+ B^\prime C\\
  =&A^\prime A +A^\prime C^\prime + B^\prime A + B^\prime C^\prime +B^\prime B+ B^\prime C
\end{align}
$$\because A A^\prime = O$$
\begin{align}
=& O +A^\prime C^\prime + B^\prime A + B^\prime C^\prime+ O+ B^\prime C\\
=& A^\prime C^\prime + B^\prime A + B^\prime(C +C^\prime)
\end{align}
$$\because A+ A^\prime = I $$
\begin{align}
   =& A^\prime C^\prime + B^\prime A + B^\prime(I)\\
   =& A^\prime C^\prime + B^\prime (A + I)
\end{align}
$$\because A+ I = I $$
\begin{align}
   =  A^\prime C^\prime + B^\prime
\end{align}

   \centering
    \begin{circuitikz}
    \draw (-1,0) to (0,0);
    \draw (0,0) to (0,1);
    \draw (0,1) [V,l = $A^\prime$] to (2,1)  ;
    \draw (2,1) [V,l = $C^\prime$] to (4,1)  ;
    \draw (0,0) to (0,-1);
    \draw (0,-1) [V,l = $B^\prime$] to (4,-1)  ;
    \draw (4,1) to (4,-1);
    \draw (4,0) to (5,0);
    \end{circuitikz}

\end{document}
